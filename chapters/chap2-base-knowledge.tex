\section{Base knowledge}
\subsection{What is a Field}
\textit{\textquotedblleft We can verify that a set is a field by checking that multiplication is a well defined operation i.e. it is independent of the representative.\textquotedblright}\\

For example, for arbitrary rational numbers $Q$:

$$
\frac{m_1}{n_1} \times \frac{p_1}{q_1}
$$

And evaluate an equivalent expresion with different representatives of the same numbers:

$$
\frac{m_2}{n_2} \times \frac{p_2}{q_2}
$$

Given that:

$$
\frac{m_1}{n_1} = \frac{m_2}{n_2} \hspace{8pt} \wedge \hspace{8pt} \frac{p_1}{q_1} = \frac{p_2}{q_2}
$$

We want to verify that both numbers (both multiplication results) are the same. To review this, we start from the tautology (intuitive truth):
\begin{align*} 
\frac{m_1}{n_1} = \frac{m_2}{n_2} &\Longleftrightarrow  m_1 \times n_2 = m_2 \times n_1 &(A) \\\\
\frac{p_1}{q_1} = \frac{p_2}{q_2} &\Longleftrightarrow  p_1 \times q_2 = p_2 \times q_1 &(B) \\
\end{align*}

Then, operating the multiplication using both representatives:

\begin{align*}
\frac{m_1}{n_1} \times \frac{p_1}{q_1} &= \frac{m_1 . p_1}{n_1 . q_1} \\\\
\frac{m_2}{n_2} \times \frac{p_2}{q_2} &= \frac{m_2 . p_2}{n_2 . q_2} \\
\end{align*}


Conveniently, we want to form $m_1 \times n_2$ to form the first ground truth.

\begin{align*}
\frac{m_1}{n_1} \times \frac{p_1}{q_1} \times n_2 &= \frac{m_1 . n_2 . p_1}{n_1 . q_1} \\\\
&= \frac{m_2 . n_1 . p_1}{n_1 . q_1} &(\text{replacing }A) \\\\
&= \frac{m_2 . p_1}{q_1} &(\text{simplifying } n_1) \\\\
\end{align*}

Applying the same logic for $p_1 \times q_2$ to form the second ground truth.

\begin{align*}
\frac{m_1}{n_1} \times \frac{p_1}{q_1} \times n_2 \times q_2 &= \frac{m_2 . p_1 . q_2}{q_1} \\\\
&= \frac{m_2 . p_2 . q_1}{q_1} &(\text{replacing }B) \\\\
&= m_2 . p_2 &(\text{simplifying } q_1) \\\\
\end{align*}

Finally, rearranging:
\begin{align*}
\frac{m_1}{n_1} \times \frac{p_1}{q_1} \times n_2 \times q_2 &= m_2 . p_2 \\\\
\frac{m_1}{n_1} \times \frac{p_1}{q_1} &= \frac{m_2 . p_2}{n_2 . q_2} \\\\
\frac{m_1}{n_1} \times \frac{p_1}{q_1} &= \frac{m_2}{n_2} \times \frac{p_2}{q_2} \hspace{5pt} \blacksquare
\end{align*}

This is not a regurous demonstration, but gives us a first step to go from the intuiton of a solution (particularly for $Q$), to a more formal procedure based on the real definition of a field.


\subsection{Formal definition}
\begin{definition}[Field]
A Field $\mathbb{F}$ is a set $A$ with two operations: addtion ($\bigoplus$) and multiplication ($\bigotimes$), with the following properties:
\begin{itemize}
\item $x, y \in \mathbb{F} \rightarrow x \bigoplus y \in \mathbb{F}$
\item $x, y \in \mathbb{F} \rightarrow x \bigoplus y = y \bigoplus x$
\item $x, y, z \in \mathbb{F} \rightarrow (x \bigoplus y) \bigoplus z = x \bigoplus (y \bigoplus z)$
\item $\exists 0 \in \mathbb{F} / \forall x \in \mathbb{F} \rightarrow x \bigoplus 0 = x$
\item $\forall x \in \mathbb{F}, \exists -x \in \mathbb{F}: x \bigoplus -x = 0$


\item $x, y \in \mathbb{F} \rightarrow x \bigotimes y \in \mathbb{F}$
\item $x, y \in \mathbb{F} \rightarrow x \bigotimes y = y \bigotimes x$
\item $x, y, z \in \mathbb{F} \rightarrow (x \bigotimes y) \bigotimes z = x \bigotimes (y \bigotimes z)$
\item $\exists 1 \in \mathbb{F} / \forall x \in \mathbb{F} \rightarrow x \bigotimes 1 = x$
\item $\forall x \in \mathbb{F} - \{0\}, \exists x^{-1} \in \mathbb{F}: x \bigotimes x^{-1} = 1$
\end{itemize}
\end{definition}

First five correspond to the addion operation, and the last five to the multiplication operation. In order to relate both sets of properties, the following axiom is stated:

\begin{axiom}[Distributive law]\label{ax:distributive}
Let $x, y, z \in \mathbb{F}$
$$
x \bigotimes (y \bigoplus z) = x \bigotimes y \bigoplus x \bigotimes z \hspace{8pt}
$$
\end{axiom}

\begin{theorem}[Zero uniqueness]
For any Field $\mathbb{F}$, there exists only one zero element.
\end{theorem}

\begin{bookproof}
Assume $0_1$ and $0_2$ are zeros for a field $\mathbb{F}$.

\begin{align*}
\forall x \in \mathbb{F}:
\begin{cases}
	0_1 + x = x \hspace{8pt} (\text{A}) \\
	0_2 + x = x \hspace{8pt} (\text{B}) \\
\end{cases}
\end{align*}
For both cases, let $x$ be $0_1, 0_2$ respectively
\begin{align*}
x = 0_2 \Rightarrow (A) \hspace{8pt} 0_1 + 0_2 = 0_2 \hspace{8pt} (\text{C}) \\\\
x = 0_1 \Rightarrow (B) \hspace{8pt} 0_2 + 0_1 = 0_1 \hspace{8pt} (\text{D}) \\\\
\Rightarrow (C) \hspace{8pt} 0_2 + 0_1 = 0_2 \text{ (Commutativity)}
\end{align*}
Comparing this result with $D$
\begin{align*}
\Rightarrow  0_2 + 0_1 = 0_2 \text{ and } 0_2 + 0_1 = 0_1\\\\
\Rightarrow  0_1 = 0_2
\end{align*}

\end{bookproof}

\subsection{Order}
\begin{definition}[Ordered sets]
A set $S$ will be ordered when having an ordering "$<$", so that $\forall x, y \in S$ one of the following properties should hold:
\begin{enumerate}
\item $x = y$
\item $x < y$
\item $y < x$
\end{enumerate}
\end{definition}

\begin{definition}[Ordered fields]\label{def:ordered-fields}
A field $\mathbb{F}$ will be ordered if it is also an ordered set. As consequence, the following properties apply:
\begin{itemize}
\item $x, y \in \mathbb{F}, x < y \rightarrow \forall z \in \mathbb{F}, x \bigoplus z <  y \bigoplus z$
\item $x, y \in \mathbb{F}, 0 \  < \ x, y \rightarrow 0 <  x \bigotimes y$
\end{itemize}
\end{definition}

\begin{theorem}
Given $\mathbb{F}$: ordered field. $x < y \wedge 0 < z \Rightarrow x \bigotimes z < y \bigotimes z$
\end{theorem}

\begin{bookproof}
Proving by contradiction, lets assume the opposite:
$$
x\bigotimes z \geq y\bigotimes z
$$
Then adding $(-x \bigotimes z)$ on both sides, we would maintain the ordering of the expression
\begin{align*}
x\bigotimes z \bigoplus (-x\bigotimes z) \geq y\bigotimes z \bigoplus (-x\bigotimes z)\\
0 \geq y\bigotimes z \bigoplus (-x\bigotimes z)
\end{align*}
Now, using \Cref{ax:distributive}, we get:
\begin{align*}
0 \geq z \bigotimes  (y \bigoplus (-x))
\end{align*}
Considering the first initial considitions for $x < y \Rightarrow y \bigoplus (-x) > 0$ and having $z > 0$, we would expect that the product these two to be $>0$ by \Cref{def:ordered-fields} (second property). Hence:
$$
0 \geq z \bigotimes  (y \bigoplus (-x)) \wedge 0 < z \bigotimes  (y \bigoplus (-x)) \rightarrow\!\leftarrow \text{(Contradiction found)}
$$
\end{bookproof}

From this point and for a better readability, + and . (or $\times$) will be used instead of $\bigoplus$ or $\bigotimes$. They will still represent the abstraction of a field's addition and product operations, without necessarily being the known addition and multiplication we could expect them to be.

\subsection{Completeness}

\newpage
