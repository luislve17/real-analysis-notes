\subsection{Completeness}

\begin{definition}[Bounds]

A set $X \subset \mathbb{F}$ ($\mathbb{F}$ ordered field) is said to be \textit{bounded above} (or respectively, \textit{bounded below}) if $\exists \ c \in \mathbb{F}$ such that $\forall a \in \mathbb{F}, a \leq x$ (or respectively, $x \leq a$). $c$ is called upper (or respectively lower) bound of $X$.

\end{definition}

\begin{definition}

A set that is both bounded above and below, is called \textit{bounded}.

\end{definition}

\begin{definition}

An element $a \in X$ is called the \textit{largest} element of $X$ if $\forall x \in X, x \leq a$. Respectively, $a \in X$ is called the \textit{smallest} element of $X$ if $\forall x \in X, a \leq x$. Simplifying the notation:

\begin{align*}
(a = max X) := (a \in X \wedge \forall x \in X, x \leq a)\\
(a = min X) := (a \in X \wedge \forall x \in X, a \leq x)
\end{align*}
\end{definition}

These read as \textit{maximal} and \textit{minimal} of $X$.
Now, given this definition, is important to notice that not every set, not even every bounded set, has a maximal or minimal element. For example:
$$
X = \{x \in \mathbb{F} \ |\  0 \leq x < 1\}
$$
Only has a minimal element ($0$), but no maximal element, since $1 \notin X$

\begin{definition}[Least Upper Bound]
The smallest $s \in X \subset \mathbb{F}$ that bounds $X$ from above is called the \textit{least upper bound} of $X$, and denoted $\sup{X}$ (read "the supremum of $X$")

\begin{align*}
(s = \sup{X}) := \forall x \in X ((x \leq s) \wedge (\forall s^{\prime} < s \  \exists x^{\prime} \in X (s^{\prime} < x^{\prime}))
\end{align*}
\end{definition}

Lets break this down by element

\begin{itemize}
\item $\forall x \in X$: The following definition applies to the whole set $X$.
\item $(x \leq s)$: Given that $s$ is an upper bound for $X$...
\item $(\forall s^{\prime} < s) \exists x^{\prime} \in X (s^{\prime} < x^{\prime})$:

\begin{itemize}
\item $(\forall s^{\prime} < s)$: Considering any arbitrary $s^{\prime}$ smaller than our upper bound $s$
\item $\exists x^{\prime} \in X$: There will be an element $x^\prime$ in $X$, so that...
\item $(s^{\prime} < x^{\prime})$: It is larger than the $s^\prime$, making $s^\prime$ to \textbf{fail} to be an upper bound.
\end{itemize}
\end{itemize}

So in summary, $s$ is $\sup{X}$ if and only if, $s$ is an upper bound, and no smaller number $s^\prime$ is an upper bound of $X$, because we can find an $x^\prime$ that is not bounded by it.

\begin{definition}[Greatest Lower Bound]
Similarly, the greatest $i \in X \subset \mathbb{F}$ that bounds $X$ below is called the \textit{greatest lower bound of X}, and denoted $\inf X$ (read "the infimum of X")

\begin{align*}
(i = \inf X) := \forall x \in X ((i \leq x) \wedge (\forall i^\prime < i \ \exists x^\prime \in X (x^\prime < i^\prime))
\end{align*}

\end{definition}

Thus, we have now the following definitions:

\begin{align*}
\sup X := \min \{c \in \mathbb{F} | \forall x \in X (x \leq c)\} \\
\inf X := \max \{c \in \mathbb{F} | \forall x \in X (c \leq x)\}
\end{align*}

It is important to note that the supremum and infimum of a set, as defined above, may not exist in an arbitrary ordered field $\mathbb{F}$. The definitions above specify what $\sup X$ and $\inf X$ mean \textit{if they exist}, but they do not guarantee existence. We will address this issue shortly.

\begin{theorem}[Uniqueness of Supremum]
Let $X \subset \mathbb{F}$ be a nonempty set in an ordered field $\mathbb{F}$. If $X$ has a supremum, then this supremum is unique.
\end{theorem}

\begin{bookproof}
Suppose $s_1$ and $s_2$ are both suprema of $X$. We will show that $s_1 = s_2$.

Since $s_1 = \sup X$, we know that $s_1$ is an upper bound of $X$. Since $s_2 = \sup X$, we know that $s_2$ is the \textit{least} upper bound of $X$. Therefore:
$$s_2 \leq s_1$$

By the same reasoning (swapping the roles of $s_1$ and $s_2$), since $s_2$ is an upper bound and $s_1$ is the least upper bound:
$$s_1 \leq s_2$$

By the antisymmetry property of order in $\mathbb{F}$, we have $s_1 \leq s_2$ and $s_2 \leq s_1$, which implies:
$$s_1 = s_2$$

Therefore, the supremum is unique.
\end{bookproof}

The proof for the uniqueness of the infimum is analogous.

\begin{example}
Consider the ordered field $\mathbb{Q}$ of rational numbers, and let:
$$S = \{x \in \mathbb{Q} \ | \ x^2 < 2\}$$

The set $S$ is nonempty (for instance, $1 \in S$) and bounded above (for instance, $2 \in \mathbb{F}$ is an upper bound). However, $S$ does not have a supremum in $\mathbb{Q}$, since if it had, it would have to equal $\sqrt{2} \notin \mathbb{Q}$, so $S$ has no least upper bound within the rational numbers.

This shows that not every ordered field has the property that bounded sets possess suprema.
\end{example}

\begin{definition}[Complete Ordered Field]
An ordered field $\mathbb{F}$ is called \textit{complete} if every nonempty subset of $\mathbb{F}$ that is bounded above has a supremum in $\mathbb{F}$.
\end{definition}

\begin{proposition}
If $\mathbb{F}$ is a complete ordered field, then every nonempty subset of $\mathbb{F}$ that is bounded below has an infimum in $\mathbb{F}$.
\end{proposition}

\begin{bookproof}
Let $X \subset \mathbb{F}$ be nonempty and bounded below. Define:
$$X' = \{-x \ | \ x \in X\}$$

\textbf{1. $X'$ is nonempty and bounded above}

Since $X$ is nonempty, $\exists x_0 \in X \Rightarrow -x_0 \in X'$ by definition of $X'$. Thus $X' \neq \emptyset$.

Now, $X$ is bounded below by the proposition, so there exists $c \in \mathbb{F}$ such that:
\begin{align*}
\forall x \in X&, \quad c \leq x \\
\forall x \in X&, \quad -x \leq -c
\end{align*}

Since $-x \in X^prime$ and $-c \in \mathbb{F}$, we can be sure that $X^prime$ is bounded above in $\mathbb{F}$.

Considering $\mathbb{F}$ is complete, and we proved $X'$ is nonempty and bounded above, $X'$ has a supremum in $\mathbb{F}$.
\vspace{.75cm}

\textbf{2. $-s$ is a lower bound for $X$.}

Let $s = \sup X'$ and $x \in X$ be arbitrary. Then $-x \in X'$ by definition of $X'$. Since $s$ is an upper bound for $X'$:

\begin{align*}
\forall -x \in X^\prime&, \quad -x \leq s \\
\Rightarrow & \quad -s \leq x
\end{align*}

Showing that $-s$ is a lower bound for $X$.
\vspace{.75cm}

\textbf{3: $-s$ is the greatest lower bound for $X$.}

Let $\ell \in \mathbb{F}$ be any lower bound for $X$.

Since $\ell$ is a lower bound for $X$:
\begin{align*}
\forall x \in X&, \quad \ell \leq x\\
\Rightarrow &\quad -x \leq -\ell
\end{align*}

Meaning that $-\ell$ is an upper bound for any $-x \in X'$. Since $s = \sup X'$, we have:
\begin{align*}
s \leq -\ell\\
\Rightarrow \ell \leq -s
\end{align*}

Showing that $-s$ is greater than or equal to every lower bound of $X$. Therefore, $-s = \inf X$.
\end{bookproof}

\begin{theorem}[Least Upper Bound Principle]
Let $\mathbb{F}$ be a complete ordered field. Every nonempty subset of $\mathbb{F}$ that is bounded above has a unique least upper bound in $\mathbb{F}$.
\end{theorem}

\begin{bookproof}
Let $X \subset \mathbb{F}$ be nonempty and bounded above.

\textbf{Existence:} Since $\mathbb{F}$ is complete, by definition of completeness, $X$ has a supremum in $\mathbb{F}$.

\textbf{Uniqueness:} By the Uniqueness of Supremum theorem, this supremum is unique.

Therefore, $X$ has a unique least upper bound.
\end{bookproof}

