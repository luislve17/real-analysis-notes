\subsection{$\mathbb{F}$ is $\mathbb{R}$}

Turns out there is a reason why it is called \textbf{\textit{Real}} Analysis. So far we have worked over a generic \textit{Complete Ordered Field} $\mathbb{F}$, that was also an \textit{Archimedean} following the expected properties. We have finally reached the point were we can state that \textbf{any} set that manages to follow all these properties is actually identical to the set of real numbers $\mathbb{R}$. In fact, $\mathbb{R}$ is the only set that accomplishes this.

Most \textit{Real Analysis} courses won't go into proving this, but as I already mentioned in the first chapter, I want to cover all doubts and missing point existent during a course like this.

We will go through the initial proofs needed to understand the main one. At the beginning its possible that some of them look unecessary or trivial, but I don't want to assume that much. Better to be safe than sorry.

\subsubsection{Isomorphism}

Our starting point will be defining what it means to have \textquotedblleft equal fields".


\begin{definition}[Field homomorphism]
Let $F, F^\prime$ fields. A function $\phi: F \rightarrow F^\prime$ is a homomorphism between $F$ and $F^\prime$ iif. it preserves the defined operations in both fields:
\begin{align*}
\phi(x + y) &= \phi(x) + \phi(y), \forall x, y \in F\\
\phi(x . y) &= \phi(x) . \phi(y), \forall x, y \in F
\end{align*}
\end{definition}

\begin{definition}[Field isomorphism]\label{def:field-isomorphism}
Let $F, F^\prime$ fields. A function $\phi: F \rightarrow F^\prime$ is a isomorphism iif. it is a homomorphism and also a bijection
\end{definition}

\begin{lemma}[Injectivity by order preservation]\label{le:inj-by-order}
Let $F, F^\prime$ fields, $\phi: F \rightarrow F^\prime$. If $\phi$ preserves order from $\mathbb{F}$:
$$x < y \Rightarrow \phi(x) < \phi(y), x, y \in F$$
Then $\phi$ is injective.
\end{lemma}

\begin{bookproof}
To establish injectivity, we must show that for any two identical images of the function $\phi$, we get that their arguments were the same, guaranteeing the uniqueness of the image:

Let $x, y \in \mathbb{F}$ so that $$\phi(x) = \phi(y)$$

By tricotomy of $F$:

Case 1: $x < y$: $$\rightarrow \phi(x) < \phi(y) (\rightarrow\leftarrow)$$
Case 2: $y < x$: $$\rightarrow \phi(y) < \phi(x) (\rightarrow\leftarrow)$$

$$\Rightarrow x = y$$

Then, $\phi$ is injective.
\end{bookproof}


\begin{lemma}[Surjectivity by construction]\label{le:sur-by-construction}
Let $F, F^\prime$ fields, $\phi: F \rightarrow F^\prime$. Define: $$\text{Im}_F := \{\phi(x), x \in F\}$$
Then, the bounded function $\phi^\prime: F \rightarrow \text{Im}_F$ is surjective
\end{lemma}

\begin{bookproof}
To prove surjectivity, we need to show that every element in $y \in \text{Im}_F$ has an $x \in F$ such that $\phi^\prime(x) = y$. Using the definition of our constructed set:
\begin{align*}
\forall y \in \text{Im}_F \Leftrightarrow y \in \{\phi(x), x \in F\}\\
\Rightarrow \exists x \in F / \forall y \in \text{Im}_F: y = \phi^\prime(x)\\
\end{align*}

Finally proving that for any image of $\phi^\prime \in \text{Im}_F$ we will have a preimage $x \in F$.

Then $\phi^\prime$ is surjective.
\end{bookproof}

On this first part, this surjectiveness by construction might be the bit that doesn't quite seem correct. Isn't it trivial to have surjectiveness if we grab the method $\phi$ and limit it to its image $\phi^\prime$?

The key here is that we first defined the set of images $\text{Im}_F$, and \textit{then} proved the surjectiveness in it. The lemma confirms that the set we defined was \textit{just right}. Not so big to leave some elements unreached, and not too small to make undefined preimages at some spots.

