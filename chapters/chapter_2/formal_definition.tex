\subsection{Formal definition}

\begin{definition}[Field]
A \textbf{field} $\mathbb{F}$ is a set with two operations: addition ($\oplus$) and multiplication ($\otimes$), with the following properties:
\begin{itemize}
\item $x, y \in \mathbb{F} \implies x \oplus y \in \mathbb{F}$
\item $x, y \in \mathbb{F} \implies x \oplus y = y \oplus x$
\item $x, y, z \in \mathbb{F} \implies (x \oplus y) \oplus z = x \oplus (y \oplus z)$
\item $\exists\, 0 \in \mathbb{F}$ such that $\forall x \in \mathbb{F}, x \oplus 0 = x$
\item $\forall x \in \mathbb{F}, \exists\, (-x) \in \mathbb{F}$ such that $x \oplus (-x) = 0$
\item $x, y \in \mathbb{F} \implies x \otimes y \in \mathbb{F}$
\item $x, y \in \mathbb{F} \implies x \otimes y = y \otimes x$
\item $x, y, z \in \mathbb{F} \implies (x \otimes y) \otimes z = x \otimes (y \otimes z)$
\item $\exists\, 1 \in \mathbb{F}$ such that $\forall x \in \mathbb{F}, x \otimes 1 = x$
\item $\forall x \in \mathbb{F} \setminus \{0\}, \exists\, x^{-1} \in \mathbb{F}$ such that $x \otimes x^{-1} = 1$
\end{itemize}
\end{definition}

The first five properties correspond to the addition operation, and the last five to the multiplication operation. In order to relate both sets of properties, the following axiom is stated:

\begin{axiom}[Distributive law]\label{ax:distributive}
Let $x, y, z \in \mathbb{F}$. Then
\[
x \otimes (y \oplus z) = x \otimes y \oplus x \otimes z
\]
\end{axiom}

\begin{theorem}[Zero uniqueness]
For any field $\mathbb{F}$, there exists only one zero element.
\end{theorem}

\begin{bookproof}
Assume $0_1$ and $0_2$ are zero elements in $\mathbb{F}$. Then
\[
\forall x \in \mathbb{F}: \quad
\begin{cases}
	0_1 + x = x & \text{(A)} \\
	0_2 + x = x & \text{(B)}
\end{cases}
\]

Setting $x = 0_2$ in A, and be $x = 0_1$ in B:
\begin{align*}
x = 0_2 \implies \text{(A)}: \quad 0_1 + 0_2 &= 0_2 \tag{C} \\
x = 0_1 \implies \text{(B)}: \quad 0_2 + 0_1 &= 0_1 \tag{D}
\end{align*}

From (C) and commutativity:
\[
0_2 + 0_1 = 0_2
\]

Comparing this result with (D):
\begin{align*}
0_2 + 0_1 = 0_2 \quad &\text{and} \quad 0_2 + 0_1 = 0_1 \\
&\implies 0_1 = 0_2
\end{align*}
\end{bookproof}

\newpage
