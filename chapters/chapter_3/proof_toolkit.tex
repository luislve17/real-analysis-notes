\newpage
\subsection*{A. Proof Techniques Toolkit}
\addcontentsline{toc}{subsection}{A. Proof Techniques Toolkit}

I wanted to take the time to review the proof techniques used per chapter. This appendix collects these with an example on each.

\subsubsection*{A.1 Archimedean Property}

\textbf{Definition:}
For any complete ordered field $F$, given any $x \in F$, there exists $n \in \mathbb{N}_F$ such that $n > x$.

\textbf{Intuition:}
No matter how large a number you pick, you can always find a natural number bigger than it. This means the natural numbers are unbounded—they grow without limit.

Used to:
\begin{itemize}
    \item Show sets are bounded (by finding an $n$ that works as an upper bound)
    \item Show sets are non-empty (by finding a negative $n$ that serves as a lower element)
    \item Construct rationals or integers around arbitrary real numbers
\end{itemize}

\textbf{Example:}

\textit{Problem:} Let $S = \{1 - \frac{1}{n} : n \in \mathbb{N}\}$. Show that $S$ is bounded above.

\textit{Solution:} By the Archimedean property, for the number $2$, there exists $n \in \mathbb{N}$ with $n > 2$. For any element $1 - \frac{1}{n} \in S$, we have 
\[1 - \frac{1}{n} < 1 < 2 < n.\]
Hence $2$ is an upper bound for $S$. \qed

\subsubsection*{A.2 Density of $\mathbb{Q}$ in $\mathbb{R}$}

\textbf{Definition:}
Between any two real numbers $\alpha < \beta$, there exists a rational number $r$ such that $\alpha < r < \beta$.

Formally: $\forall \alpha, \beta \in \mathbb{R}$ with $\alpha < \beta$, $\exists r \in \mathbb{Q}$ such that $\alpha < r < \beta$.

\textbf{Intuition:}
The rationals are ``everywhere'' in the reals—no matter how close two real numbers are, you can always squeeze a rational between them. This allows us to:
\begin{itemize}
    \item Approximate irrational numbers by rational numbers
    \item Find elements ``as close as we want'' to a target value
    \item Bridge constructions between $\mathbb{Q}$ and $\mathbb{R}$
\end{itemize}

\textbf{Example:}

\textit{Problem:} Show that between $\sqrt{2}$ and $\sqrt{3}$, there exists a rational number.

\textit{Solution:} We have $\sqrt{2} \approx 1.414\ldots$ and $\sqrt{3} \approx 1.732\ldots$. Take $r = \frac{3}{2} = 1.5$. Since $1.414 < 1.5 < 1.732$, we have 
\[\sqrt{2} < \frac{3}{2} < \sqrt{3},\]
and $\frac{3}{2} \in \mathbb{Q}$. \qed

\subsubsection*{A.3 Proof by Contradiction}

\textbf{Definition:}
To prove a statement $P$, assume $\neg P$ and derive a contradiction. Since assuming $\neg P$ leads to something impossible, $P$ must be true.

\textbf{Intuition:}
``If assuming the opposite leads to nonsense, then the original must be true.'' This technique is especially powerful when:
\begin{itemize}
    \item Direct proof is difficult or unclear
    \item You want to prove something doesn't exist
    \item You want to prove a uniqueness property (by assuming two things exist and deriving a contradiction)
\end{itemize}

\textbf{Example:}

\textit{Problem:} Prove that $\sqrt{2}$ is irrational.

\textit{Solution:} Suppose $\sqrt{2} \in \mathbb{Q}$. Then $\sqrt{2} = \frac{p}{q}$ where $p, q \in \mathbb{Z}$, $q \neq 0$, and $\gcd(p,q) = 1$ (in lowest terms). 

Squaring both sides: $2 = \frac{p^2}{q^2}$, so $p^2 = 2q^2$. 

Thus $p^2$ is even, which implies $p$ is even (if $p$ were odd, $p^2$ would be odd). Write $p = 2k$ for some $k \in \mathbb{Z}$. 

Then $(2k)^2 = 2q^2$, so $4k^2 = 2q^2$, hence $q^2 = 2k^2$. 

Thus $q^2$ is even, so $q$ is even. 

But then both $p$ and $q$ are even, contradicting $\gcd(p,q) = 1$. 

Therefore $\sqrt{2} \notin \mathbb{Q}$. \qed

\subsubsection*{A.4 Mathematical Induction}

\textbf{Definition:}
To prove a statement $P(n)$ holds for all $n \in \mathbb{N}$:
\begin{enumerate}
    \item \textit{Base case:} Show $P(1)$ is true
    \item \textit{Inductive step:} Assume $P(k)$ is true (inductive hypothesis), then prove $P(k+1)$ is true
\end{enumerate}

\textbf{Intuition:}
Like dominoes—if the first one falls (base case) and each falling domino knocks over the next (inductive step), then all dominoes fall. We use induction for:
\begin{itemize}
    \item Constructing $\mathbb{N}_F$ step-by-step
    \item Proving properties that hold ``for all natural numbers''
    \item Building sequences or recursive structures
\end{itemize}

\textbf{Example:}

\textit{Problem:} Prove that for all $n \in \mathbb{N}$, the sum $1 + 2 + 3 + \cdots + n = \frac{n(n+1)}{2}$.

\textit{Solution:} 

\textit{Base case ($n=1$):} $\text{LHS} = 1$, $\text{RHS} = \frac{1(2)}{2} = 1$. \checkmark

\textit{Inductive step:} Assume the formula holds for $n = k$, i.e., 
\[1 + 2 + \cdots + k = \frac{k(k+1)}{2}.\]
We must show it holds for $n = k+1$:
\begin{align*}
1 + 2 + \cdots + k + (k+1) &= \frac{k(k+1)}{2} + (k+1) \\
&= (k+1)\left[\frac{k}{2} + 1\right] \\
&= (k+1)\cdot\frac{k+2}{2} \\
&= \frac{(k+1)(k+2)}{2}
\end{align*}

This is exactly the formula for $n = k+1$. By induction, the formula holds for all $n \in \mathbb{N}$. \qed

\subsubsection*{A.5 Supremum Arguments}

\textbf{Definition:}
To prove $s = \sup(S)$, we must show:
\begin{enumerate}
    \item $s$ is an upper bound: $\forall x \in S$, $x \leq s$
    \item $s$ is the \textit{least} upper bound: $\forall \varepsilon > 0$, $\exists x \in S$ such that $s - \varepsilon < x \leq s$
\end{enumerate}

Equivalently for (2): For any $s' < s$, there exists $x \in S$ with $x > s'$ (so $s'$ is not an upper bound).

\textbf{Intuition:}
The supremum is the ``ceiling'' of a set—nothing in the set exceeds it, but you can get arbitrarily close to it from below. This technique is used constantly to:
\begin{itemize}
    \item Define functions via suprema of rational approximations
    \item Show that bounds are tight (no smaller bound works)
    \item Bridge rational approximations to real limits
\end{itemize}

\textbf{Example:}

\textit{Problem:} Let $S = \{1 - \frac{1}{n} : n \in \mathbb{N}\}$. Prove that $\sup(S) = 1$.

\textit{Solution:} 

1. \textit{Upper bound:} For any $n \in \mathbb{N}$, we have $\frac{1}{n} > 0$, so $1 - \frac{1}{n} < 1$. Thus $1$ is an upper bound for $S$.

2. \textit{Least upper bound:} Let $\varepsilon > 0$. By the Archimedean property, there exists $N \in \mathbb{N}$ with $N > \frac{1}{\varepsilon}$, so $\frac{1}{N} < \varepsilon$. 

Then $1 - \frac{1}{N} \in S$ and 
\[1 - \left(1 - \frac{1}{N}\right) = \frac{1}{N} < \varepsilon,\]
meaning $1 - \varepsilon < 1 - \frac{1}{N} \leq 1$. 

So within any $\varepsilon$-neighborhood of $1$, there exists an element of $S$. Thus $1$ is the least upper bound. \qed

\subsubsection*{A.6 Trichotomy in Ordered Fields}

\textbf{Definition:}
In an ordered field $F$, for any $x, y \in F$, exactly one of the following holds:
\begin{enumerate}
    \item $x = y$
    \item $x < y$  
    \item $x > y$
\end{enumerate}

\textbf{Intuition:}
Any two elements are either equal or one is strictly larger—no other options exist. This fundamental property is used in:
\begin{itemize}
    \item Proof by cases (exhaustively considering all three possibilities)
    \item Proving injectivity (if $f(x) = f(y)$, then $x$ cannot be $<$ or $>$ than $y$)
    \item Establishing contradictions in order-preservation arguments
\end{itemize}

\textbf{Example:}

\textit{Problem:} Prove that if $x^2 = y^2$ in an ordered field $F$, then $x = y$ or $x = -y$.

\textit{Solution:} We have $x^2 - y^2 = 0$, so $(x-y)(x+y) = 0$. 

In a field, a product equals zero if and only if at least one factor equals zero. Thus $x - y = 0$ or $x + y = 0$.

\begin{itemize}
    \item If $x - y = 0$, then $x = y$.
    \item If $x + y = 0$, then $x = -y$.
\end{itemize}

By trichotomy, these are the only possibilities (we cannot have both $x - y \neq 0$ and $x + y \neq 0$ simultaneously if $x^2 = y^2$). \qed

\vspace{1em}
