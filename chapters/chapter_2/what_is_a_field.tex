\subsection{What is a Field}

\textit{``To verify that a set forms a field, we check that multiplication is well-defined. That is, independent of the choice of representative.''}\\

For example, for arbitrary rational numbers $Q$:
\[
\frac{m_1}{n_1} \times \frac{p_1}{q_1}
\]

And evaluate an equivalent expression with different representatives of the same numbers:
\[
\frac{m_2}{n_2} \times \frac{p_2}{q_2}
\]

Given that:
\[
\frac{m_1}{n_1} = \frac{m_2}{n_2} \quad \text{and} \quad \frac{p_1}{q_1} = \frac{p_2}{q_2}
\]

We wish to show that both products coincide. To review this, we start from the tautology (intuitive truth):
\begin{align*} 
\frac{m_1}{n_1} = \frac{m_2}{n_2} &\iff m_1 \times n_2 = m_2 \times n_1 \tag{A} \\
\frac{p_1}{q_1} = \frac{p_2}{q_2} &\iff p_1 \times q_2 = p_2 \times q_1 \tag{B}
\end{align*}

Then, operating the multiplication using both representatives:
\begin{align*}
\frac{m_1}{n_1} \times \frac{p_1}{q_1} &= \frac{m_1 \cdot p_1}{n_1 \cdot q_1} \\
\frac{m_2}{n_2} \times \frac{p_2}{q_2} &= \frac{m_2 \cdot p_2}{n_2 \cdot q_2}
\end{align*}

Conveniently, we want to form $m_1 \times n_2$ to use the first ground truth:
\begin{align*}
\frac{m_1}{n_1} \times \frac{p_1}{q_1} \times n_2 &= \frac{m_1 \cdot n_2 \cdot p_1}{n_1 \cdot q_1} \\
&= \frac{m_2 \cdot n_1 \cdot p_1}{n_1 \cdot q_1} \quad \text{(replacing using A)} \\
&= \frac{m_2 \cdot p_1}{q_1} \quad \text{(simplifying $n_1$)}
\end{align*}

Applying the same logic for $p_1 \times q_2$ to use the second ground truth:
\begin{align*}
\frac{m_1}{n_1} \times \frac{p_1}{q_1} \times n_2 \times q_2 &= \frac{m_2 \cdot p_1 \cdot q_2}{q_1} \\
&= \frac{m_2 \cdot p_2 \cdot q_1}{q_1} \quad \text{(replacing using B)} \\
&= m_2 \cdot p_2 \quad \text{(simplifying $q_1$)}
\end{align*}

Finally, rearranging:
\begin{align*}
\frac{m_1}{n_1} \times \frac{p_1}{q_1} \times n_2 \times q_2 &= m_2 \cdot p_2 \\
\frac{m_1}{n_1} \times \frac{p_1}{q_1} &= \frac{m_2 \cdot p_2}{n_2 \cdot q_2} \\
\frac{m_1}{n_1} \times \frac{p_1}{q_1} &= \frac{m_2}{n_2} \times \frac{p_2}{q_2} \qquad \square
\end{align*}

While not a rigorous proof, this gives us a first step to go from the intuition of a solution (particularly for $\mathbb{Q}$) to a more formal procedure based on the real definition of a field.

\newpage
