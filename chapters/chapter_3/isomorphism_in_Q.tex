\subsection{Isomorphism in $\mathbb{Q}s$}
\begin{definition}[$\phi$ function]
Let $\phi: \mathbb{Q_F} \rightarrow \mathbb{Q}$ defined as:
$$\phi(\frac{m}{n}) := \frac{m_\mathbb{F}}{n_\mathbb{F}}$$

Where:
\begin{itemize}
\item $m, n \in \mathbb{Q} \ (m \in \mathbb{Z}, n \in \mathbb{Z^+})$
\item $m_\mathbb{F}, n_\mathbb{F} \in \mathbb{Q_F}$
\end{itemize}
\end{definition}

\begin{lemma}\label{le:phi-homomorphism}
$\phi: \mathbb{Q_F} \rightarrow \mathbb{Q}$
$$\phi(\frac{m}{n}) := \frac{m_\mathbb{F}}{n_\mathbb{F}}$$ is an homomorphism
\end{lemma}

\begin{bookproof}
For $m_1/n_1, m_2/n_2 \in \mathbb{Q}$

\begin{enumerate}
\item $\phi(x + y) = \phi(x) + \phi(y)$
\begin{align*}
\phi(m_1/n_1 + m_2/n_2) &= \phi(\frac{m_1.n_2 + m_2.n_1}{n_1.n_2}) \\
&= \frac{(m_1.n_2 + m_2.n_1)_\mathbb{F}}{(n_1.n_2)_\mathbb{F}} \\ 
&= \frac{(m_1.n_2)_\mathbb{F} + (m_2.n_1)_\mathbb{F}}{(n_1)_\mathbb{F}.(n_2)_\mathbb{F}} \\
&= \frac{(m_1)_\mathbb{F}.(n_2)_\mathbb{F} + (m_2)_\mathbb{F}.(n_1)_\mathbb{F}}{(n_1)_\mathbb{F}.(n_2)_\mathbb{F}} \\
&= \frac{(m_1)_\mathbb{F}}{(n_1)_\mathbb{F}} + \frac{(m_2)_\mathbb{F}}{(n_2)_\mathbb{F}} \\\\
\phi(m_1/n_1 + m_2/n_2) &= \phi(m_1/n_1) + \phi(m_2/n_2)
\end{align*}

\item $\phi(x.y) = \phi(x) . \phi(y)$
\begin{align*}
\phi(m_1/n_1 . m_2/n_2) &= \phi(\frac{m_1.m_2}{n_1.n_2}) \\
&= \frac{(m_1.m_2)_\mathbb{F}}{(n_1.n_2)_\mathbb{F}} \\
&= \frac{(m_1)_\mathbb{F}.(m_2)_\mathbb{F}}{(n_1)_\mathbb{F}.(n_2)_\mathbb{F}} \\
&= \frac{(m_1)_\mathbb{F}}{(n_1)_\mathbb{F}} . \frac{(m_2)_\mathbb{F}}{(n_2)_\mathbb{F}} \\\\
\phi(m_1/n_1 . m_2/n_2) &= \phi(m_1/n_1) . \phi(m_2/n_2)
\end{align*}
\end{enumerate}

Finally, $\phi$ is an homomorphism.
\end{bookproof}

Now comes the monster. In order to get the isomorphism, we need to make sure $\phi$ preserves order to prove injectivity by \cref{le:inj-by-order}, and this step alone takes several previous lemmas to cover it fully, and of course we will see them all. Just remember we are having fun.

\begin{lemma}[Ordering in $\mathbb{N_F}$]
For $n, k \in \mathbb{N}, n_\mathbb{F}, k_\mathbb{F} \in \mathbb{N_F}$
\begin{enumerate}
\item If $n < k \rightarrow n_\mathbb{F}< k_\mathbb{F}$
\item If $n = k \rightarrow n_\mathbb{F}= k_\mathbb{F}$
\item If $n > k \rightarrow n_\mathbb{F}> k_\mathbb{F}$
\end{enumerate}
\end{lemma}

\begin{bookproof}
\begin{enumerate}
\item $n = k \rightarrow n_\mathbb{F} = k_\mathbb{F}$

$$n_\mathbb{F} := \sum_{1}^{n}{1_F} = \sum_{1}^{k}{1_F} = n_\mathbb{F}$$

\item $n < k \rightarrow n_\mathbb{F} < k_\mathbb{F}$

$$n < k \Leftrightarrow \exists p \in \mathbb{N}: n + p = k$$

Now, $k_\mathbb{F}$ can be expressed as:
$$k_\mathbb{F} := \sum_{1}^{k}{1_F} = \sum_{1}^{n}{1_F} + \sum_{1}^{p}{1_F}  = n_\mathbb{F} + p_\mathbb{F}$$

Since $p > 1$, using order preservation in addition for fields (\cref{def:ordered-fields}) we have:
$$k_\mathbb{F} = n_\mathbb{F} + p_\mathbb{F} > n_\mathbb{F} + 0_\mathbb{F} > n_\mathbb{F}$$

\item $n > k \rightarrow n_\mathbb{F} > k_\mathbb{F}$

$$n > k \leftrightarrow k < n$$
From 2.
$$n_\mathbb{n} > k_\mathbb{n} $$
\end{enumerate}
\end{bookproof}

\begin{lemma}[Ordering in $\mathbb{Z_F}$]\label{le:ordering-Z}
For $n, k \in \mathbb{Z}, n_\mathbb{F}, k_\mathbb{F} \in \mathbb{Z_F}$
\begin{enumerate}
\item If $n < k \rightarrow n_\mathbb{F}< k_\mathbb{F}$
\item If $n = k \rightarrow n_\mathbb{F}= k_\mathbb{F}$
\item If $n > k \rightarrow n_\mathbb{F}> k_\mathbb{F}$
\end{enumerate}
\end{lemma}

\begin{bookproof}
\begin{enumerate}
\item $n = k \rightarrow n_\mathbb{F} = k_\mathbb{F}$
\begin{enumerate}
\item $n, k >= 0$: Same as the proof for $\mathbb{N_F}$
\item $n, k < 0$: $$n = k \leftrightarrow -k = -n \rightarrow -k, -n \in \mathbb{N}$$
\end{enumerate}
We now proceed exactly as the proof for $\mathbb{N_F}$

\item $n < k \rightarrow n_\mathbb{F} < k_\mathbb{F}$, similarly to $\mathbb{N_F}$:
\begin{align*}
n < k \Leftrightarrow \exists p \in \mathbb{N}: n + p = k
\end{align*}

\begin{enumerate}
\item $n >= 0$: Since $n \in \mathbb{N} \rightarrow (n + p)_\mathbb{F} = n_\mathbb{F} + p_\mathbb{F}$

\item $n < 0$: Labeling $n = -t$ just to indicate it is negative by sign:
\begin{align*}
&\rightarrow n = -t \leftrightarrow n_\mathbb{F} = -t_\mathbb{F} \wedge (-t) + p = k \\
&\rightarrow p = k + t \\
&\rightarrow p_\mathbb{F} = k_\mathbb{F} + t_\mathbb{F} \\
&\rightarrow p_\mathbb{F} = k_\mathbb{F} - n_\mathbb{F} \\
&\rightarrow n_\mathbb{F} + p_\mathbb{F} = k_\mathbb{F} \\
&\rightarrow n_\mathbb{F} + p_\mathbb{F} = k_\mathbb{F} = (n + p)_\mathbb{F}
\end{align*}
Therefore, we prooved for any case of $n, k \in \mathbb{Z}: n < k \rightarrow n_\mathbb{F} < k_\mathbb{F}$
\end{enumerate}

\item $n > k \rightarrow n_\mathbb{F}> k_\mathbb{F}$
\\\\
Labeling acrodingly: $n > k \leftrightarrow -n^\prime > -k^\prime \leftrightarrow n^\prime < k^\prime (n^\prime, k^\prime \in \mathbb{N})$\\
With this we just proceed as in 2
\end{enumerate}
\end{bookproof}

So far so \textquotedblleft good\textquotedblright. Now, before moving to $\mathbb{Q}$ we need an extra tool for operating multiplications in $\mathbb{Z_F}$.

\begin{lemma}[Multiplication in $\mathbb{Z_F}$]\label{le:mult-in-Z}
For any $n, k \in \mathbb{Z}$ \\
$$(n.k)_\mathbb{F} = n_\mathbb{F} . k_\mathbb{F} \in \mathbb{F}$$
\end{lemma}

\begin{bookproof}
\begin{enumerate}
\item $n, k > 0:$\\\\
We have:
$$(n . k)_\mathbb{F} = \sum_{1}^{n.k} 1_\mathbb{F} = \sum_{1}^{n} \sum_{1}^{k} 1_\mathbb{F} =  \sum_{1}^{n} k_\mathbb{F} = n_\mathbb{F} . k_\mathbb{F}$$
Which is just adding the additive neutral $n.k$ times.


\item $n > 0, k < 0:$ \\\\
Writing $k$ as $-m$ to denote negativeness, we have:
$$(n.k)_\mathbb{F} = (n(-m))_\mathbb{F} = (-n.m)_\mathbb{F} = -(n.m)_\mathbb{F}$$
From 1:
$$(n.k)_\mathbb{F} = -n_\mathbb{F}.m_\mathbb{F} = n_\mathbb{F}.-m_\mathbb{F} = n_\mathbb{F}.k_\mathbb{F}$$
\end{enumerate}
\end{bookproof}

\begin{lemma}[Ordering in $\mathbb{Q}$]\label{le:ordering-Q}
Let $\frac{m_1}{n_1}, \frac{m_2}{n_2} \in \mathbb{Q}$ \\
$$\frac{m_1}{n_1} < \frac{m_2}{n_2} \rightarrow (\frac{m_1}{n_1})_\mathbb{F} < (\frac{m_2}{n_2})_\mathbb{F}$$
\end{lemma}

\begin{bookproof}
Arbitrary locking $n_1, n_2 \in \mathbb{N}$ without losing generalization, we have:
$$\frac{m_1}{n_1} < \frac{m_2}{n_2} \leftrightarrow m_1.n_2 < m_2.n_1$$
Maintaing the inequality direction
Now, from \cref{le:ordering-Z}:
$$m_1.n_2 < m_2.n_1 \leftrightarrow (m_1.n_2)_\mathbb{F} < (m_2.n_1)_\mathbb{F}$$
Next, from \cref{le:mult-in-Z}:
$$(m_1.n_2)_\mathbb{F} < (m_2.n_1)_\mathbb{F} \leftrightarrow (m_1)_\mathbb{F}.(n_2)_\mathbb{F} < (m_2)_\mathbb{F}.(n_1)_\mathbb{F}$$
Now, since we fixed $n1, n2 \in \mathbb{N}$, we can use $(n1.n2)_\mathbb{F}$ to divide the expression:
$$(m_1)_\mathbb{F}.(n_2)_\mathbb{F} < (m_2)_\mathbb{F}.(n_1)_\mathbb{F} \leftrightarrow \frac{(m_1)_\mathbb{F}.(n_2)_\mathbb{F}}{(n_1.n_2)_\mathbb{F}} . \frac{(m_2)_\mathbb{F}.(n_1)_\mathbb{F}}{(n_1.n_2)_\mathbb{F}}$$
Next, we just distribute and simplify:
$$\frac{(m_1)_\mathbb{F}.(n_2)_\mathbb{F}}{(n_1.n_2)_\mathbb{F}} . \frac{(m_2)_\mathbb{F}.(n_1)_\mathbb{F}}{(n_1.n_2)_\mathbb{F}} \leftrightarrow \frac{(m_1)_\mathbb{F}}{(n_1)_\mathbb{F}} . \frac{(m_2)_\mathbb{F}}{(n_1)_\mathbb{F}}$$

Finally, we have:
$$\frac{m_1}{n_1} < \frac{m_2}{n_2} \leftrightarrow \frac{(m_1)_\mathbb{F}}{(n_1)_\mathbb{F}} . \frac{(m_2)_\mathbb{F}}{(n_1)_\mathbb{F}}$$
\\
$$\frac{m_1}{n_1} < \frac{m_2}{n_2} \leftrightarrow (\frac{m_1}{n_1})_\mathbb{F} < (\frac{m_2}{n_2})_\mathbb{F}$$
\end{bookproof}

Now if we recall, we did all of this as a previous step for proving the injectivity of $\phi$. We can make it happen now.

\begin{lemma}\label{le:phi-injective}
$\phi: \mathbb{Q_F} \rightarrow \mathbb{Q}$
$$\phi(\frac{m}{n}) := \frac{m_\mathbb{F}}{n_\mathbb{F}}$$ is injective
\end{lemma}

\begin{bookproof}
Let $\frac{m_1}{n_1}, \frac{m_2}{n_2} \in \mathbb{Q}$, such that:

$$\frac{m_1}{n_1} < \frac{m_2}{n_2}$$

By \cref{le:ordering-Q}, we have:

$$\frac{m_1}{n_1} < \frac{m_2}{n_2} \rightarrow (\frac{m_1}{n_1})_\mathbb{F} < (\frac{m_2}{n_2})_\mathbb{F}$$

By definition of $\phi$, we can replace:

\begin{align*}
\frac{m_1}{n_1} < \frac{m_2}{n_2} &\rightarrow \frac{(m_1)_\mathbb{F}}{(n_1)_\mathbb{F}} < \frac{(m_2)_\mathbb{F}}{(n_2)_\mathbb{F}} \\\\
\frac{m_1}{n_1} < \frac{m_2}{n_2} &\rightarrow \phi(\frac{m_1}{n_1}) < \phi(\frac{m_1}{n_1}) \\
\end{align*}
$$\Rightarrow \phi : \mathbb{Q} \rightarrow \mathbb{Q_F} \text{ preserves order.}$$ \\

Finally, from \cref{le:inj-by-order} (Injectivity by order preservation) we assert:\\

$$\Rightarrow \phi : \mathbb{Q} \rightarrow \mathbb{Q_F} \text{ is injective.}$$
\end{bookproof}

\begin{lemma}\label{le:phi-surjective}
$\phi: \mathbb{Q_F} \rightarrow \mathbb{Q}$
$$\phi(\frac{m}{n}) := \frac{m_\mathbb{F}}{n_\mathbb{F}}$$ is surjective
\end{lemma}

\begin{bookproof}
By set construction (\cref{le:sur-by-construction}) we defined $\phi: \mathbb{Q_F} \rightarrow \mathbb{Q}$, meaning that:
$$\mathbb{Q} = Im(\mathbb{Q_F})$$
Then
$$\phi \text{ is surjective.}$$
\end{bookproof}

This completes the construction.

\begin{lemma}
$\phi: \mathbb{Q_F} \rightarrow \mathbb{Q}$
$$\phi(\frac{m}{n}) := \frac{m_\mathbb{F}}{n_\mathbb{F}}$$ is an isomorphism
\end{lemma}

\begin{bookproof}
\begin{enumerate}
\item By \cref{le:phi-homomorphism}, $\phi$ is an homomorphism.
\item By \cref{le:phi-injective} and \cref{le:phi-surjective}, $\phi$ is bijective
\end{enumerate}
Then, by \cref{def:field-isomorphism}

$$\Rightarrow \phi \text{ is an isomorphism}.$$
\end{bookproof}

Securing this is a big step, and recalling our strategy we now need to use this isomorphism $\phi$ in $\mathbb{Q_F} \rightarrow \mathbb{Q}$, and somehow transform it into a $\Phi$ that extends this to the whole $\mathbb{F} \rightarrow \mathbb{R}$.

Although this has being a long ride, personally I don't think this was a complex construction, or a hard-to-follow implementation of $\phi$. Everything just landed in place.

The tricky part begins when we introduce $\Phi$ and try to understand what it really is. Still, I wouldn’t include anything that I wouldn’t find clear or satisfactory from a student’s perspective (keep in mind I'm also learning while writing). If something feels confusing or lacks intuition, I prefer to leave it out entirely and try again.
