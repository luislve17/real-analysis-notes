\subsection{Archimedean property}

Now that we achieved completeness on an ordered field, we can now extend the definitions and properties into a theorem/property that uses the result from the \textit{Least Upper Bound principle} (\cref{th:lub-principle}).

\begin{theorem}\label{th:archimedean}
% ∀x∈F,∃n∈N such that n>x
Let $\mathbb{F}$ be a complete ordered field, then $$\forall x \in \mathbb{F}, \exists \ n \in \mathbb{N} \subset \mathbb{F}: n > x$$
\end{theorem}

\begin{bookproof}
We first define the subset $\mathbb{N} \subset \mathbb{F}$ \textbf{inductively}, to work with a generalized version of the set of \textquotedblleft natural numbers"
\begin{enumerate}
\item $1_{\mathbb{F}}$ is the multiplicative identity in $\mathbb{F}$
\item $n + 1_{\mathbb{F}} \in \mathbb{N}, \forall n \in \mathbb{N}$ defines the induction step over $\mathbb{N}$, using the addition operation from the field $\mathbb{F}$
\end{enumerate}

We proceed by contradiction. Suppose $\exists x \in \mathbb{F}, \forall n \in \mathbb{N}:$
$$
n \leq x
$$

Meaning that $\mathbb{N}$ is bounded above. Now, since $\mathbb{N}$ is not empty, and is bounded by an element of the (complete ordered) field that contains it ($x \in \mathbb{F}$), then by \textit{L.U.B. principle} (\cref{th:lub-principle}), $\mathbb{N}$ must have a supremum.
\\\\
Let $s := \sup{\mathbb{N}} \in \mathbb{F}$ be that supremum of $\mathbb{N}$. Then, since $s$ is the least upper bound, any element that is less than $s$ will no longer be an upper bound for $\mathbb{N}$. Conveniently we take $s - 1_{\mathbb{F}}$:
\begin{align*}
\exists n_0 \in \mathbb{N}:& s - 1_{\mathbb{F}} < n_0\\
&\Rightarrow s < n_0 + 1_{\mathbb{F}}
\end{align*}

But by our inductive definition of $\mathbb{N}: (n_0 + 1_{\mathbb{F}}) \in \mathbb{N}$. So we found out that $s$ ($\sup{\mathbb{N}}$) fails to be an upper bound for an element in $\mathbb{N}$.
\\\\
Therefore, our original assumption of $x$ must be wrong.
\end{bookproof}

Now, is important to notice that $\mathbb{F}$ implies an archimedien field, but the opposite won't necessarily be true.

\begin{example}
Using an Archimedean field, that is not complete, let us take $\mathbb{F} = \mathbb{Q}$
$$x \in \mathbb{Q} \wedge n \in \mathbb{N} \subset \mathbb{Q}$$
It is true that, for any $x$ there is a natural number $n$ so that $$n>x$$But invoking the previous example (\cref{ex:s-in-q-not-bounded}) we can take a subset of $\mathbb{Q}$ that is not bounded in $\mathbb{Q}$, failing to properly form a complete set.
\end{example}

Closing this chapter, let us emphazise the fact that the construction of $\mathbb{N} \subset \mathbb{F}$ by induction was crucial to generalize the property for any complete ordered field. Next we will explain why this wasn't necessary at all.

